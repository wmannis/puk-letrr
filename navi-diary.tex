\documentclass[nofonts]{tufte-handout}
\usepackage{fontspec}
\usepackage{xunicode}
\usepackage{xltxtra}
\defaultfontfeatures{Ligatures=TeX}
%\setromanfont{Linux Libertine O}
\setmainfont[Scale=1.25]{Brill}

\usepackage{ulem}

\definecolor{navi}{rgb}{0.0, 0.45, 0.55}
\newcommand{\N}[1]{\textbf{\textcolor{navi}{#1}}}

% To work around a known bug.
% https://tex.stackexchange.com/questions/200722/xetex-seems-to-break-headers-in-tufte-handout
\ifxetex
  \newcommand{\textls}[2][5]{%
    \begingroup\addfontfeatures{LetterSpace=#1}#2\endgroup
  }
  \renewcommand{\allcapsspacing}[1]{\textls[15]{#1}}
  \renewcommand{\smallcapsspacing}[1]{\textls[10]{#1}}
  \renewcommand{\allcaps}[1]{\textls[15]{\MakeTextUppercase{#1}}}
  \renewcommand{\smallcaps}[1]{\smallcapsspacing{\scshape\MakeTextLowercase{#1}}}
  \renewcommand{\textsc}[1]{\smallcapsspacing{\textsmallcaps{#1}}}
\fi

\usepackage{hyperref}
\hypersetup{bookmarksopen=false,
  pdfpagemode=UseNone,
  colorlinks=true,
  urlcolor=blue,
  linkcolor=black,    % no links for footnotes; URLs will still have color
  pdftitle={Using a Diary to Practice Na'vi},
  pdfauthor={William S. Annis},
  pdfkeywords={Na'vi, grammar, xenolinguistics, xenoethology}
}


\title{Using a Diary to Practice Na'vi}
\author{Wm S. Annis}

\begin{document}
\maketitle%

When learning a new language, practice is the name of the
game.\marginnote{``Repetition, repetition.''}  But we don't always
have time or opportunities to practice a new language as often as we'd
like, and with a constructed language like Na'vi there are even more
difficulties.  First, most of our training materials are generated in
the community, so we have to rely on the free time they have.  Second,
the number of speakers is quite small — compared to, say, French — so
not only are there only a few people to practice with, there aren't
Na'vi language TV shows coming out every week, and there are only a
few projects to do, say, the news, etc.  \marginnote{There are some
  occasional projects to do regular blogs, but, again, that relies on
  the continuing efforts of volunteers.

  Volunteers are a vital part of the Na'vi language community, but
  they also have lives to live apart from Na'vi.  We can't expect
  them to do everything forever.  Sometimes we need to improvise.}

One technique I've used in the last few years for both learning and
developing my own constructed languages, is keeping a diary.  But this
is an \textit{excruciatingly boring diary.}  When I start a new
language, I am really only talking about the weather — ``it rained
today; it was windy; the sun was out,'' etc.  At the start, I am
absolutely not trying to discuss the political situation in
Kazakhstan, say, or doing a critical analysis of the video game
\textit{Animal Crossing}.  By focusing on just very simple topics I
know I can do \textit{a little} language practice every day, a much
better approach than trying to gulp down a lot of the language in
intense, irregular bursts.  And we know writing things down by
hand—but not typing them—is an aid to memory.

In addition to the daily production practice, you can go back and
read ealier entries for comprehension practice.  Over time, you can
also start to see your improvements in the language.

\bigskip
\N{Ayutralìri ayrìk sngä'i zivup.}\marginnote{\N{ay-}:
  plural; \N{\uline{ut}ral}: ``tree;'' \N{-ìri} topical;
  \N{rìk}: ``leaf;'' \N{\uline{sngä}'i} ``begin;'' \N{zup}:
  ``fall;'' \N{‹iv›}: subjunctive.  \textit{The tree leaves
    start to fall.}}
\bigskip

Another thing you can do with a diary, which is related to at least
part of the appeal of \textit{Avatar,} is practice amateur ethology —
record the environment, weather, plants, and animals of your local
area.  You can look up from your studies, the zillions of distractions
and rage harvests of the internet, and get outside for a few minutes to
collect your thoughts and observations.  
\marginnote{\textbf{Do not treat this diary as a holy relic!}  Cross
  things out.  Scribble in the margins.  Draw a weird bug.  Maybe
  scribble all over the first page, just to get past the urge to
  protect the nice clean pages.}
I repeat from earlier, this does \textit{not} have to be deep thoughts
or elaborate description.  The point is to make using the language
daily easier.  As you get more comfortable, you can start adding more
and more to what you say in the diary.

\vfill
    {\footnotesize \hrule width1cm
     \vskip1ex
      \noindent This work is licensed under the Creative Commons 4.0 \href{https://creativecommons.org/licenses/by-nc-sa/4.0/}{BY-NC-SA} license.}

\section*{Starting Out}

You can't necessarily start your diary on Day One of your Na'vi
studies.  It will help to have a little pronunciation practice, with
live people if at all possible (there are at least two Discord groups
that can help with this, as of Fall 2022, advertised in the usual
places). \marginnote{The underlined syllable is where the stress
  accent goes, used to introduce new words.} You should know the
convention of why I underline parts of words when I define them in the
notes.  But it is in no way necessary for you to have mastered the
grammar at this point. \marginnote{I'm going to wave my hands for some
  things from here out about the grammar, giving you set patterns to
  just memorize rather than big grammatical explanations.  For
  beginners, this is a normal part of language instruction.}  A little
conversational knowledge, and just the start of some grammar is
enough.  The diary itself can act as a grammar training ground.

If you find yourself agonizing over a diary entry, \textit{stop!}
It's not yet time to say that in your diary.  The point of this is to
get quick practice, not to cause yourself grammar and vocabulary
trauma.  Save the sentence, and maybe workshop it with
\N{ay\uline{lì'}fyao\uline{lo'}tu} on a Na'vi language forum or in
Discord.  Don't cause yourself stress.

\subsection*{The Basics}
\N{Taw lu lepwopx.}\marginnote{\N{taw}: ``sky;'' \N{lu}: ``be;''
  \N{lep\uline{wopx}}: ``cloudy.''  \textit{It's cloudy.}}

\bigskip
Na'vi grammar is flexible enough that we don't have to worry about its
slightly more complex tense marking so long as the time is clear.  So
it's useful to have a few time words at hand, like \N{set} ``now,''
\N{\uline{ye'}rìn} ``soon,'' \N{\uline{maw}krr} ``afterwards, later,''
\N{fì\uline{trr}} ``today,'' \N{trr\uline{am}} ``yesterday,''
\N{trr\uline{ay}} ``tomorrow,'' and \N{nì\uline{mun}} ``again.''

In English, some unspecified ``it'' does the weather: ``it's raining,
it's cold, it's cloudy.''  Na'vi does things a bit differently.
\marginnote{Consider practicing the pronunciation of all this
vocabulary with people on Discord.}

For the condition of the sky, \N{taw} ``sky'' does most of the work:

\begin{quotation}
\noindent\N{Taw lu lep\uline{wopx}.} ``It's cloudy'' (the sky is cloudy).\\
\noindent\N{Taw lu \uline{pi}ak.} ``It's clear'' (the sky is open). \\
\noindent\N{Taw lu tstu.} ``It's completely overcast'' (the sky is closed).\\
\noindent\N{\uline{Tsaw}ke \uline{lrr}tok si.} ``The sun is shining'' (the sun smiles).
\end{quotation}

In Na'vi, precipitation is not something that ``it'' does, rather the
motion of the precipitation is described:

\begin{quotation}
\noindent\N{\uline{Tom}pa zup.} ``It's raining'' (rain falls).\\
\noindent\N{\uline{Her}wì zup.} ``It's snowing'' (snow falls).
\end{quotation}

And similarly, temperature is something \N{ya} ``air'' is:

\begin{quotation}
\noindent\N{Ya lu wew.} ``It's cold'' (the air is cold). \\
\noindent\N{Ya lu wur.} ``It's cool'' (the air is cool). \\
\noindent\N{Ya lu sang.} ``It's warm'' (the air is warm). \\
\noindent\N{Ya lu som.} ``It's hot'' (the air is hot).
\end{quotation}

Any of these statements can be negaged by putting \N{ke} \textit{not}
\marginnote{\N{Ke} gets the main phrase stress of the \N{ke} + verb
  phrase: \N{\uline{ke} zup} not \N{ke \uline{zup}}.}
before the verb (\N{lu} or \N{zup}): as in \N{tompa ke zup} ``it
didn't rain,'' \N{ye'rìn ya ke lu wew} ``soon it will not be cold.''
With verbal expressions formed with \N{si}, the negation goes before
\N{si}, as in \N{tsawke lrrtok ke si} ``the sun is not shining.''

Finally, if you're feeling extravagant, you can join statements with
\N{\uline{ul}te} ``and'' or \N{slä} ``but.''  This will let you say
things like \N{trram taw lu lepwopx, slä tompa ke zup} ``yesterday it
was cloudy, but it didn't rain.''

Work with just this vocabulary for a few weeks, up to a month, to get
practice.  \marginnote{Just write dates as you would in your native
  language for now.  Na'vi numbers and time-keeping would only
  complicate things horribly.}  Hopefully you'll have a few changes in
weather.  Remember you have \N{ke} ``not,'' and you can use that to
add a little variation to your statements.  If you are very new to
Na'vi, don't try to skip ahead!  Start simple, and work your way up to
using more of \N{ke} ``not,'' \N{ulte} ``and,'' and \N{slä} ``but.''

Try to write something every day, but if you forget, don't worry about
it too much.  After all, you do have \N{trram} ``yesterday'' there to
work with.

After a few weeks, come back and read the next section.


\subsection*{More Weather}
You have several words for the temperature you've been practicing a
while now.  There are a few more options available:

\begin{quotation}
\noindent\N{Ya lu \uline{txa}wew.} ``It's very cold.''\\
\noindent\N{Ya lu \uline{tsya}fe.} ``It's mild, comfortable.'' \\
\noindent\N{Ya lu \uline{txa}som.} ``It's very hot.''
\end{quotation}

\noindent Notice how the \N{txa-} element is prefixed to the words you
already know for ``cold'' and ``hot'' to create the more intense
versions.

Another way to modify the temperature words is with adverbs of degree,
like \N{nì\uline{txan}} ``very, a lot,'' and \N{nì\uline{'it}} ``a
bit, a little.''  \marginnote{There is a tendency to use the pattern
  [\textsc{adj} \N{lu nìtxan/nì'it}/etc.], contrary to the expected English word
  order.  A word order like \N{ya sang lu nì'it} for ``it's a little
  warm'' is seen often on Frommer's blog.  But \N{ya lu nì'it sang}
  is fine, too.}
For example, \N{fìtrr ya wur lu nìtxan} ``it is very cool today.''
You can also modify \N{lepwopx} ``cloudy'' with \N{nì\uline{hol}}
``few, not many'' for lightly cloudy conditions, and
\N{nì\uline{pxay}} ``many'' for a heavily clouded sky, as in \N{taw
  lepwopx lu nìpxay} ``there are many clouds.''

Next, let's talk about the wind (\N{\uline{hu}fwe})!  In Na'vi the
wind \N{tì\uline{ran}} ``walks'' if it's a gentle wind, and \N{tul}
``runs'' if it is windier.

\begin{quotation}
\noindent\N{Tìran hufwe.} ``It's breezy.'' \\
\noindent\N{Tul hufwe nì\uline{win}.} ``It's very windy'' (lit., ``the wind runs quickly'')
\end{quotation}

When you've used these weather phrases for a few weeks,
\marginnote{Fortunately, most of us don't have to deal with
  \N{\uline{tskxay}tsyìp} ``hail'' very often, though that will make
  it a bit harder to learn this term just through the diary method.}
do take a look at Paul Frommer's two blog posts about the weather,
\href{https://naviteri.org/2011/04/yafkeykiri-plltxe-frapo-everyone-talks-about-the-weather/}{Everyone
  talks about the weather} and
\href{https://naviteri.org/2011/05/weather-part-2-and-a-bit-more-2/}{Weather
  Part 2}.  This will give you more vocabulary to work with.

\section*{Narrative}
In normal conversation—and writing—we do not just emit independent
sentences.\marginnote{I assume you're not only using this document to
  learn Na'vi, and that you have learned a few more things by now,
  including the pronouns, how to form plurals, and some additional,
  non-weather vocabulary.}  We join sentence to sentence, to show how
events related to each other.  This is a major part of using language
fluently.  The weather is perfect for practicing this.

When you don't say what you wanted to say, you might be tempted to
cross things out and start again.  That's fine, but you can also
correct yourself in Na'vi, using a word like \N{ko\uline{lan}} ``I
mean, rather,'' to keep the flow.

\subsection*{A Little Tense}
Na'vi verbs change tense with infixes—they shove syllables into the
middle of the verb.  This is not a familiar process for most people,
and though it seems strage at first, it's not too difficult to pick up
with a little practice.

Rather than explain the whole process here, I'm going to just give you
a few verbs we've learned already with a few of the tense infixes for
you to pick and choose as you need. 

\medskip
\begin{tabular}{llll}
  & Imperfective & Past & Future \\
\N{lu} ``be'' & \N{le\uline{ru}} ``is/was being'' & \N{la\uline{mu}}
    ``was'' & \N{la\uline{yu}} ``will be'' \\
\N{zup} ``fall'' & \N{ze\uline{rup}} ``is/was falling'' & \N{za\uline{mup}}
    ``fell'' & \N{za\uline{yup}} ``will fall''\\
\N{lrrtok si} ``smile, `shine''' & \N{lrrtok se\uline{ri}} &
    \N{lrrtok sa\uline{mi}} & \N{lrrtok sa\uline{yi}} \\
\N{tì\uline{ran}} ``walk'' & \N{terì\uline{ran}} & \N{tamì\uline{ran}} &
    \N{tayì\uline{ran}} \\
\N{tul} ``run'' & \N{te\uline{rul}} & \N{ta\uline{mul}}
     & \N{ta\uline{yul}} 
\end{tabular}
\medskip

\noindent The first time you \textit{see} the infixing,
\marginnote{There are several more
  options for the infixes, but this small set, along with the time
  adverbs, should keep you going for a while.}
it can seem pretty wild.  But if you pronounce the examples above,
with the correct stress, you can see that it's often quite easy to
hear what's going on, more so than just seeing it.

The conjunction \N{teng\uline{krr}} ``while, at the same time'' is
nearly always used with the imperfective, and can be used in lots of
ways to say more complex things.

\begin{quotation}
\noindent\N{Tengkrr zerup tompa, hufwe tul.} ``While it was raining,
the wind was strong.''\\

\noindent\N{Tengkrr tsawke lrrtok seri, ya lu sang.} ``While the sun is
shining it is warm.''
\end{quotation}

\noindent But you can also use the imperfective without a conjunction,
\N{Set zerup herwì} ``it is snowing now.''

\subsection*{Word Order}
While I have not done anything really exotic with the word order in my
examples so far, it is true that Na'vi has an extremely flexible word
order.  The example above, ``it is snowing,'' could plausibly be
expressed with all the following phrasings:

\begin{quotation}
  \noindent\N{Set zerup herwì.} \\
  \noindent\N{Set herwì zerup.} \\
  \noindent\N{Zerup set herwì.} \\
  \noindent\N{Zerup herwì set.} \\
  \noindent\N{Herwì set zerup.} \\
  \noindent\N{Herwì zerup set.}
\end{quotation}

\noindent In general, the end of the sentence is where Na'vi puts
focused, emphasized elements.  So, if you wanted to express that the
snow was the unexpected element of the sentence, you'd use one of the
orders that puts \N{herwì} at the end. \marginnote{In English we
  signal older information with the word ``the.''} Older information
tends to go first, so if you were already talking about snow, or if it
was completely expected, you'd use a word order putting \N{herwì}
first.

These are general tendencies, and once again I encourage you not to
get too hung up on picking your word orders.  In some sense, the word
order rules of Na'vi are still being discovered.  But because
different speakers of Na'vi, both in the fandom and in the movies,
will use various word orders, it is good to get practice with
different options.


\section*{Ethology}
As a scientific practice, ethology is the study of animal behavior in
the animal's natural context and environment.  Amateur ethology is an
excellent diary pratice when you're learning a new language, and it
helps to connect you to the environment you live in.  We've already
been tracking the weather.  Now we can add animals.

\N{I\uline{o}ang} ``animal,'' \N{\uline{ya}yo} ``bird,'' \N{'eng}
``beak,'' \N{tsyal} ``wing,'' \N{\uline{hì}'ang} ``insect,''
\N{\uline{li}ni} ``young of an animal or insect,'' \N{pa\uline{yo}ang}
``fish.''  Animal activities: \N{\uline{tswa}yon} ``fly,'' \N{spä}
``jump,'' \N{frìp} ``bite,'' \N{sngap} ``sting,'' \N{\uline{ta}ron}
``hunt,'' \N{nong\uline{spe'}} ``pursue with intent to capture.''

One problem you will run into is how to name earth critters in Na'vi.
\marginnote{This is another time when it would be useful to workshop
  with other Na'vi speakers.}
One way out is to use the convention of adding the diminutive ending,
\N{-tsyìp}, to Pandoran animals that approximate Earth animals, such
as \N{palu\uline{lu}kantsyìp} ``little Thanator'' for earth ``cat.''
Another dodge is to try to adapt the local name to the Na'vi noun
system, just for the purposes of your diary, when no Pandoran animal
seems a likely match (which will be often).  For example,
\marginnote{That asterisk, \N{*skìwrrlì}, means I made this word up.}
\N{*skìwrrlì} for ``squirrel.'' 

Another possibility is to come up with phrases.  For example, if I
wanted to talk about fireflies, I can start with \N{hì'ang}
``insect.''  The Na'vi word \N{tan\uline{hì}} ``star'' also refers to
the bioluminescent freckles many Pandoran creatures have.  So, you
could use \N{hì'ang a hu sanhì} ``bug with stars.''  Or, you could use
the verb \N{nrr} ``glow,'' for \N{hì'ang a nrr} ``bug which glows.''

It might be useful now to learn \href{https://naviteri.org/2010/08/mipa-ayopin-mipa-ayliu-new-colors-new-words/}{Na'vi colors}, too.

\section*{Keep on Keeping on}

Other things you can write about as you get better with the grammar
and learn more words: what you did at work or school; how you're
feeling; \marginnote{I have allergies.  \N{Oeri menari krro fkxake}
(\N{oe}: ``I;'' \N{-(ì)ri}: topical; \N{me-}: dual; \N{\uline{na}ri} ``eye;''
  \N{krro}: ``sometimes;'' \N{\uline{fkxa}ke}: ``itch'').
  \textit{Sometimes my eyes itch}.}
what you ate; what you did with friends (or who you're
feuding with); what you are reading, watching, listening to; world
events, if you really must; pet antics; etc., etc., etc.

When you expand topics, it's good to focus on that topic for a while,
to burn in vocabulary.  If you start talking about your cat, stick to
that — and the weather, of course — for a few weeks.  Then add the
next thing.  You don't have to follow this mechanically — a little
extra spice is nice from time to time.  But the goal is always to make
this a non-stressful learning tool.

Even beyond the names for Earth critters, sometimes there will not be
Na'vi words for what you want to say.  \marginnote{Na'vi has no word
  for ``sandal.''  How might you say that?  \N{Hawn\uline{ven}
    a\uline{pi}ak} ``open shoe?'' \N{Hawn\uline{ven} a fa se\uline{lem}}
  ``shoe with cords?''  Consider the possibility that just \N{hawnven}
  is fine.}  Finding ways to talk around those gaps is a
necessary skill for using Na'vi, and good practice in itself.

Try to say first what you plan to write, to get more speaking
practice.

Remember, writing something very simple every day matters more than
writing a lot once a week.  Even if your local weather is boring, 
\marginnote{If you live in some place like southern California, 
  you might need to branch out \textit{a bit} beyond the weather
  sooner than others will.}
and you just write \N{tsawke lrrtok si} \textit{the sun shines} a lot,
that's still practice.  Go back and read an older entry or two, and
come back the next day with just a little more.


\bigskip
\N{Eywa ngahu.}


\end{document}

