\documentclass[nofonts]{tufte-handout}
\usepackage{fontspec}
\usepackage{xunicode}
\usepackage{xltxtra}
\defaultfontfeatures{Ligatures=TeX}
%\setromanfont{Linux Libertine O}
\setmainfont[Scale=1.2]{Brill}

\usepackage{ulem}

\definecolor{navi}{rgb}{0.0, 0.45, 0.55}
\newcommand{\N}[1]{\textbf{\textcolor{navi}{#1}}}

% To work around a known bug.
% https://tex.stackexchange.com/questions/200722/xetex-seems-to-break-headers-in-tufte-handout
\ifxetex
  \newcommand{\textls}[2][5]{%
    \begingroup\addfontfeatures{LetterSpace=#1}#2\endgroup
  }
  \renewcommand{\allcapsspacing}[1]{\textls[15]{#1}}
  \renewcommand{\smallcapsspacing}[1]{\textls[10]{#1}}
  \renewcommand{\allcaps}[1]{\textls[15]{\MakeTextUppercase{#1}}}
  \renewcommand{\smallcaps}[1]{\smallcapsspacing{\scshape\MakeTextLowercase{#1}}}
  \renewcommand{\textsc}[1]{\smallcapsspacing{\textsmallcaps{#1}}}
\fi

\title{Using a Diary to Practice Na'vi}
\author{Wm S. Annis}

\begin{document}
\maketitle%

When learning a new language, practice is the name of the game.  But,
we don't always have time or opportunities to practice a new language
as often as we'd like, and with a constructed language like Na'vi
there are even more difficulties.  First, all our training materials
are generated in the community, so we have to rely on the free time
they have.  Second, the number of speakers is quite small — compared
to, say, French — so not only are there only a few people to practice
with, there aren't Na'vi-language TV shows coming out every week, only
a few projects to do, say, the news, etc.  \marginnote{There are some
  occasional projects to do regular blogs, but, again, that relies on
  the continuing efforts of volunteers.

  Volunteers are a vital part of the Na'vi language community, but
  they also have lives to live apart from Na'vi.  We can't expect
  them to do everything forever.  Sometimes we need to improvise.}

One technique I've used in the last few years for both learning and
developing my own constructed languages, is a keeping diary.  But this
is an \textbf{excruciatingly boring diary.}  When I start a new
language, I am really only talking about the weather — ``it rained
today; it was windy; the sun was out,'' etc.  At the start, I am
absolutely not trying to discuss the political situation in
Kazakhstan, say, or doing a critical analysis of the video game
\textit{Animal Crossing}.  By focusing on just very simple topics I
know I can do \textit{a little} language practice every day, a much
better approach than trying to gulp down a lot of the language in
irreguar, intense bursts.

In addition to the daily production practice, you can go back and
ready ealier entries for comprehension practice.  Over time, you can
also start to see your improvements in the language.

\bigskip
\N{Ayutralìri ayrìk sngä'i zivup.}\marginnote{\N{ay-}:
  plural; \N{utral}: ``tree;'' \N{-ìri} topical;
  \N{rìk}: ``leaf;'' \N{sngä'i} ``begin;'' \N{zup}:
  ``fall;'' \N{‹iv›}: subjunctive.  \textit{The tree leaves
    started to fall.}}
\bigskip

Another thing you can do with a diary, which is related to at least
part of the appeal of \textit{Avatar,} is practice amateur ethology —
record the environment, weather, plants, and animals of your local
area.  You can look up from your studies, the zillions of distractions
and rage harvests of the internet, get outside for a few minutes to
collect your thoughts and observations.  I repeat from earlier, this
does \textbf{not} have to be deep thoughts or elaborate description.
The point is to make using the language daily easier.  As you get more
comfortable, you start adding more and more to what you say in the
diary. 

\section*{Starting Out}

You can't necessarily start your diary on Day One of your Na'vi
studies.  It will help to have a little pronunciation


\section*{Sequencing}


\section*{Keep on Keeping on}

Other things you can write about as you get better with the grammar
and learn more words: what you did at work or school; how you're
feeling; \marginnote{I have allergies.  \N{Oeri menari krro fkxake}
(\N{oe}: ``I;'' \N{-(ì)ri}: topical; \N{me-}: dual; \N{\uline{na}ri} ``eye;''
  \N{krro}: ``sometimes;'' \N{\uline{fkxa}ke}: ``itch'').
  \textit{Sometimes my eyes itch}.}
what you ate; what you did with friends (or who you're
feuding with); what you are reading, watching, listening to; world
events, if you really must; etc., etc., etc.

Sometimes there will not be Na'vi words for what you want to say.
Finding ways to talk around those gaps is a necessary skill for using
Na'vi, and good practice in itself.

Try to say first what you plan to write, to get more speaking
practice.

Remember, writing something very simple every day matters more than
writing a lot once a week.  Even if your local weather is boring, 
\marginnote{If you live in some place like southern California, 
  you might need to branch out \textit{a bit} beyond the weather
  sooner than others will.}
and you just write \N{tsawke lrrtok si} \textit{the sun shines} a lot,
that's still practice.  Go back and read an older entry or two, and
come back the next day with just a little more.


\bigskip
\N{Eywa ngahu.}

\end{document}

