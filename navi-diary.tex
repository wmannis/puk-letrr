\documentclass[nofonts]{tufte-handout}
\usepackage{fontspec}
\usepackage{xunicode}
\usepackage{xltxtra}
\defaultfontfeatures{Ligatures=TeX}
%\setromanfont{Linux Libertine O}
\setmainfont[Scale=1.25]{Brill}

\usepackage{ulem}

\definecolor{navi}{rgb}{0.0, 0.45, 0.55}
\newcommand{\N}[1]{\textbf{\textcolor{navi}{#1}}}

% To work around a known bug.
% https://tex.stackexchange.com/questions/200722/xetex-seems-to-break-headers-in-tufte-handout
\ifxetex
  \newcommand{\textls}[2][5]{%
    \begingroup\addfontfeatures{LetterSpace=#1}#2\endgroup
  }
  \renewcommand{\allcapsspacing}[1]{\textls[15]{#1}}
  \renewcommand{\smallcapsspacing}[1]{\textls[10]{#1}}
  \renewcommand{\allcaps}[1]{\textls[15]{\MakeTextUppercase{#1}}}
  \renewcommand{\smallcaps}[1]{\smallcapsspacing{\scshape\MakeTextLowercase{#1}}}
  \renewcommand{\textsc}[1]{\smallcapsspacing{\textsmallcaps{#1}}}
\fi

\usepackage{hyperref}
\hypersetup{bookmarksopen=false,
  pdfpagemode=UseNone,
  colorlinks=true,
  urlcolor=blue,
  linkcolor=black,    % no links for footnotes; URLs will still have color
  pdftitle={Using a Diary to Practice Na'vi},
  pdfauthor={William S. Annis},
  pdfkeywords={Na'vi, grammar, xenolinguistics, xenoethology}
}


\title{Using a Diary to Practice Na'vi}
\author{Wm S. Annis}

\begin{document}
\maketitle%

When learning a new language, practice is the name of the
game.\marginnote{``Repetition, repetition.''}  But, we don't always
have time or opportunities to practice a new language as often as we'd
like, and with a constructed language like Na'vi there are even more
difficulties.  First, most of our training materials are generated in
the community, so we have to rely on the free time they have.  Second,
the number of speakers is quite small — compared to, say, French — so
not only are there only a few people to practice with, there aren't
Na'vi language TV shows coming out every week, only a few projects to
do, say, the news, etc.  \marginnote{There are some occasional
  projects to do regular blogs, but, again, that relies on the
  continuing efforts of volunteers.

  Volunteers are a vital part of the Na'vi language community, but
  they also have lives to live apart from Na'vi.  We can't expect
  them to do everything forever.  Sometimes we need to improvise.}

One technique I've used in the last few years for both learning and
developing my own constructed languages, is a keeping diary.  But this
is an \textbf{excruciatingly boring diary.}  When I start a new
language, I am really only talking about the weather — ``it rained
today; it was windy; the sun was out,'' etc.  At the start, I am
absolutely not trying to discuss the political situation in
Kazakhstan, say, or doing a critical analysis of the video game
\textit{Animal Crossing}.  By focusing on just very simple topics I
know I can do \textit{a little} language practice every day, a much
better approach than trying to gulp down a lot of the language in
irreguar, intense bursts.  And we know writing things down is an aid
to memory.

In addition to the daily production practice, you can go back and
read ealier entries for comprehension practice.  Over time, you can
also start to see your improvements in the language.

\bigskip
\N{Ayutralìri ayrìk sngä'i zivup.}\marginnote{\N{ay-}:
  plural; \N{\uline{ut}ral}: ``tree;'' \N{-ìri} topical;
  \N{rìk}: ``leaf;'' \N{\uline{sngä}'i} ``begin;'' \N{zup}:
  ``fall;'' \N{‹iv›}: subjunctive.  \textit{The tree leaves
    started to fall.}}
\bigskip

Another thing you can do with a diary, which is related to at least
part of the appeal of \textit{Avatar,} is practice amateur ethology —
record the environment, weather, plants, and animals of your local
area.  You can look up from your studies, the zillions of distractions
and rage harvests of the internet, get outside for a few minutes to
collect your thoughts and observations.  I repeat from earlier, this
does \textbf{not} have to be deep thoughts or elaborate description.
The point is to make using the language daily easier.  As you get more
comfortable, you start adding more and more to what you say in the
diary. 


\section*{Starting Out}

You can't necessarily start your diary on Day One of your Na'vi
studies.  It will help to have a little pronunciation practice, with
live people if at all possible (there are at least two Discord groups
that can help with this, as of Fall 2022, advertised in the usual
places).  You should know the convention of why I underline parts of
words when I define them in the notes.\marginnote{It's where the
  stress accent goes, used to introduce new words.}  But it is in no
way necessary for you to have mastered the grammar at this
point. \marginnote{I'm going to wave my hands for some things from
  here out about the grammar, giving you set patterns to just memorize
  rather than big grammatical explanations.  For beginners, this is a
  normal part of language instruction.}  A little conversational
knowledge, and just the start of some grammar is enough.  The diary
itself can act as a grammar training ground.

If you find yourself agonizing over a diary entry, \textbf{stop!}
It's not yet time to say that in your diary.  The point of this is to
get quick practice, not to cause grammar and vocabulary trauma.  Save
the sentence, and maybe workshop it with
\N{ay\uline{lì'}fyao\uline{lo'}tu} on a Na'vi language forum or in
Discord.  Don't cause yourself stress.

\subsection*{The Basics}
\N{Taw lu lepwopx.}\marginnote{\N{taw}: ``sky;'' \N{lu}: ``be;''
  \N{lep\uline{wopx}}: ``cloudy.''  \textit{It's cloudy.}}

\bigskip
Na'vi grammar is flexible enough that we don't have to worry about its
slightly more complex tense marking so long as things are clear.  So
it's useful to have a few time words at hand, like \N{set} ``now,''
\N{\uline{ye'}rìn} ``soon,'' \N{\uline{maw}krr} ``afterwards, later,''
\N{fì\uline{trr}} ``today,'' \N{trr\uline{am}} ``yesterday,''
\N{trr\uline{ay}} ``tomorrow.''

In English, some unspecified ``it'' does the weather: ``it's raining,
it's cold, it's cloudy.''  Na'vi does things a bit differently.

For the condition of the sky, \N{taw} ``sky'' does most of the work:

\begin{quotation}
\noindent\N{Taw lu lep\uline{wopx}.} ``It's cloudy'' (the sky is cloudy).\\
\noindent\N{Taw lu \uline{pi}ak.} ``It's clear'' (the sky is open). \\
\noindent\N{Taw lu tstu.} ``It's completely overcast'' (the sky is closed).
\end{quotation}

In Na'vi, precipitation is not something that ``it'' does, rather its
motion is described: 

\begin{quotation}
\noindent\N{\uline{Tom}pa zup.} ``It's raining'' (rain falls).\\
\noindent\N{\uline{Her}wì zup.} ``It's snowing'' (snow falls).
\end{quotation}

And similarly, temperature is something \N{ya} ``air'' is:

\begin{quotation}
\noindent\N{Ya lu wew.} ``It's cold'' (the air is cold). \\
\noindent\N{Ya lu wur.} ``It's cool'' (the air is cool). \\
\noindent\N{Ya lu sang.} ``It's warm'' (the air is warm). \\
\noindent\N{Ya lu som.} ``It's hot'' (the air is hot).
\end{quotation}

Any of these statements can be negaged by putting \N{ke} \textit{not}
before the verb (\N{lu} or \N{zup}): as in \N{tompa ke zup} ``it
didn't rain,'' \N{ye'rìn ya ke lu wew} ``soon it will not be cold.''

Finally, if you're feeling extravagant, you can join statements with
\N{\uline{ul}te} ``and'' or \N{slä} ``but.''  This will let you say
things like \N{trram taw lu lepwopx slä tompa ke zup} ``yesterday it
was cloudy, but it didn't rain.''

Work with just this vocabulary for a few weeks, up to a month, to get
practice.  \marginnote{Just write dates as you would in your native
  language for now.  Na'vi numbers and time-keeping would only
  complicate things horribly.}  Hopefully you'll have a few changes in
weather.  Remember you have \N{ke} ``not,'' and you can use that to
add a little variation to your statements.  If you are very new to
Na'vi, don't try to skip ahead!  Start simple, and work your way up to
using more of \N{ke} ``not,'' \N{ulte} ``and,'' and \N{slä}.

Try to write something every day, but if you forget, don't worry about
it too much.  After all, you do have \N{trram} ``yesterday'' there to
work with.

After a few weeks, come back and read the next section.


\subsection*{Weather Exotica}

Next, let's talk about the wind!


When you've used these weather phrases for a few weeks, do take a look
at Paul Frommer's two blog posts about the weather,
  \href{https://naviteri.org/2011/04/yafkeykiri-plltxe-frapo-everyone-talks-about-the-weather/}{Everyone talks about the weather} and
  \href{https://naviteri.org/2011/05/weather-part-2-and-a-bit-more-2/}{Weather Part 2}.
This will give you more vocabulary to work
with. \marginnote{Fortunately, most of us don't have to deal with
  \N{\uline{tskxay}tsyìp} ``hail'' very often, though that will make it
    a bit harder to learn this term just through the diary method.}


\section*{Narrative}
In normal conversation — and writing — we do not just emit independent
sentences.  We join sentence to sentence, to show how events related
to each other.  This is a major part of using language fluently.  The
weather is perfect for practicing this.

Don't overthink it — Na'vi word order is quite flexible — but keep in
mind that oomph goes at the end.  

(And \N{nìtxan}, \N{nì'it}).

\subsection*{A Little Tense}


\section*{Keep on Keeping on}

Other things you can write about as you get better with the grammar
and learn more words: what you did at work or school; how you're
feeling; \marginnote{I have allergies.  \N{Oeri menari krro fkxake}
(\N{oe}: ``I;'' \N{-(ì)ri}: topical; \N{me-}: dual; \N{\uline{na}ri} ``eye;''
  \N{krro}: ``sometimes;'' \N{\uline{fkxa}ke}: ``itch'').
  \textit{Sometimes my eyes itch}.}
what you ate; what you did with friends (or who you're
feuding with); what you are reading, watching, listening to; world
events, if you really must; pet antics; etc., etc., etc.

When you expand topics, it's good to focus on that topic for a while,
to burn in vocabulary.  If you start talking about your cat, stick to
that — and the weather, of course — for a few weeks.  Then add the
next thing.  You don't have to follow this mechanically, a little
extra spice is nice from time to time.  But the goal is always to make
this a non-stressful learning tool.

Sometimes there will not be Na'vi words for what you want to say.
Finding ways to talk around those gaps is a necessary skill for using
Na'vi, and good practice in itself.

Try to say first what you plan to write, to get more speaking
practice.

Remember, writing something very simple every day matters more than
writing a lot once a week.  Even if your local weather is boring, 
\marginnote{If you live in some place like southern California, 
  you might need to branch out \textit{a bit} beyond the weather
  sooner than others will.}
and you just write \N{tsawke lrrtok si} \textit{the sun shines} a lot,
that's still practice.  Go back and read an older entry or two, and
come back the next day with just a little more.


\bigskip
\N{Eywa ngahu.}

\end{document}

